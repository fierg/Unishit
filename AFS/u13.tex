\documentclass[10pt, a4paper]{article}
\usepackage{amssymb}
\usepackage[utf8]{inputenc}
\usepackage[]{csquotes}
\usepackage[german]{babel}
\usepackage{graphicx}
\usepackage{amsmath,wasysym}
\usepackage{a4wide}


\begin{document}


\section*{13.Übung}
\paragraph{Sven Fiergolla 1252732}
\subsection*{Aufgabe a)}

-

\subsection*{Aufgabe b)}

\setlength{\parindent}{0pt}
\paragraph{}
Vorgehensweise der TM $M$ und der UTM $U$:\\
Eingabe:\par

\paragraph{}
\begin{small}
\raisebox{\depth}{\rotatebox{180}{ee}};$\Square D\Square A\Square D\Square D\Square C\Square R\Square D\Square A\Square A\Square ;\Square \Square D\Square A\Square A\Square D\Square D\Square C\Square C\Square R\Square D\Square A\Square ::\Square $\\
\end{small}\par

\paragraph{}
Übersetzung der Eingabe nach: (aktueller Zustand, gelesenes Zeichen, zu schreibendes Zeichen, Kopfbewegung, neuer Zustand)\\
$\Rightarrow$ $;(s_1,\Square ,a_1,R,s_2);(s_2,\Square ,a_2,R,s_1)::$\\
die zu simulierende TM  $M$ schreibt also bei $\Square$ als Eingabe eine endlose Folge von $a_1,a_2,a,1,a_2,\dots$ bzw. $0,1,0,1,0,\dots$ auf das Band.\par
\medskip
Ablauf UTM $U_M$
Beginn in Zustand $q_0$ $\rightarrow$ $anf$ \\ \medskip
Maschine $U_M$ schreibt $::\Square \underbrace{ : \Square D \Square A }$ \par


\paragraph{}
\begin{small}
\raisebox{\depth}{\rotatebox{180}{ee}};$\Square D\Square A\Square D\Square D\Square C\Square R\Square D\Square A\Square A\Square ;\Square \Square D\Square A\Square A\Square D\Square D\Square C\Square C\Square R\Square D\Square A\Square ::\Square : \Square D \Square A \Square$\\
\end{small}\par

\paragraph{}
Zustand $anf$ $\rightarrow$ $kom$\\
Maschine geht zur linkesten Konfiguration und markiert nach dem folgenden : die Eingabe und den Anfangszustand mit y\\
\begin{small}
\par
\paragraph{}
\raisebox{\depth}{\rotatebox{180}{ee}};$\Square D \Square  A \Square  D \Square  D \Square  C \Square  R \Square  D \Square  A \Square  A \Square  ;\Square \Square D\Square A\Square A\Square D\Square D\Square C\Square C\Square R \Square D\Square A\Square ::\Square : \Square D y A y $
\end{small}\par

\paragraph{}
Zustand $kom$ $\rightarrow$ $kmp$\\
Maschine geht zur rechtesten, nicht mit z markierte Konfiguration und markiert diese mit x\\
\begin{small}
\par
\paragraph{}
\raisebox{\depth}{\rotatebox{180}{ee}};$\Square D \Square  A \Square  D \Square  D \Square  C \Square  R \Square  D \Square  A \Square  A \Square  ; z  \Square D x  A x  A x  D x  D \Square  C \Square  C \Square  R \Square  D \Square  A  \Square  ::\Square : \Square D y A y$
\end{small}\par

\paragraph{}
Zustand $kmp$ $\rightarrow$ $anf$\\
Maschine vergleicht mit x und y markierte Konfigurationen. Diese sind ungleich, folglich werden die x entfernt und die nächste Konfiguration wird ausprobiert.\\
\begin{small}
\par
\paragraph{}
\raisebox{\depth}{\rotatebox{180}{ee}};$\Square D \Square  A \Square  D \Square  D \Square  C \Square  R \Square  D \Square  A \Square  A \Square  ; z  \Square D \Square  A \Square  A \Square  D \Square  D \Square  C \Square  C \Square  R \Square  D \Square  A  \Square  ::\Square : \Square D y A y$
\end{small}\par

\paragraph{}
Zustand $anf \rightarrow kom \rightarrow kmp$\\
Nächste Instruktion wird mit x markiert.\\
\begin{small}
\par
\paragraph{}
\raisebox{\depth}{\rotatebox{180}{ee}};$z D x  A x  D x  D \Square  C \Square  R \Square  D \Square  A \Square  A \Square  ; z  \Square D \Square  A \Square  A \Square  D \Square  D \Square  C \Square  C \Square  R \Square  D \Square  A  \Square  ::\Square : \Square D y A y$
\end{small}\par


\paragraph{}
Zustand $kmp \rightarrow s_1$\\
Maschine vergleicht neue mit x und y markierte Konfigurationen. Diese sind gleich, folglich wird in $s_1$ gewechselt und der Lesekopf auf : platziert.\\
\begin{small}
\par
\paragraph{}
\raisebox{\depth}{\rotatebox{180}{ee}};$z D x  A x  D x  D \Square  C \Square  R \Square  D \Square  A \Square  A \Square  ; z  \Square D \Square  A \Square  A \Square  D \Square  D \Square  C \Square  C \Square  R \Square  D \Square  A  \Square  ::\Square : \Square D y A y$
\end{small}\par

\paragraph{}
Zustand $s_1 \rightarrow s_2 \rightarrow mf_1$\\
Maschine markiert auszuführende Operationen\\

\begin{small}
\par
\paragraph{}
\raisebox{\depth}{\rotatebox{180}{ee}};$z D x  A x  D x  D u  C u R u  D y  A y  A y ; z  \Square D \Square  A \Square  A \Square  D \Square  D \Square  C \Square  C \Square  R \Square  D \Square  A  \Square  ::\Square : \Square D y A y$
\end{small}\par

\paragraph{}
Zustand $mf_1 \rightarrow mf_2 \rightarrow mf_3 \rightarrow mf_5 \rightarrow sh_1$\\
Maschine markiert letzte Instruktion um sie im folgenden als vollständige Instruktions ans Ende des Bands zu schreiben.\\

\begin{small}
\par
\paragraph{}
\raisebox{\depth}{\rotatebox{180}{ee}};$z D x  A x  D x  D u  C u R u  D y  A y  A y ; z  \Square D v  A v  A v  D v  D x  C x  C x  R w  D w  A  w  :: \Square : \Square D y A y$
\end{small}\par

\paragraph{}
Zustand $sh_1\rightarrow sh_2 \rightarrow sh_3 \rightarrow inst$\\
Maschine entscheidet über Ausgabe. Je nach R/L andere Reihenfolge der Ausgabe der Konfiguration.\\
\textit{Anmerkung: \enquote{zwischen die vollständigen Konfigurationen} unklare Formulierung}\\


\begin{small}
\par
\paragraph{}
\raisebox{\depth}{\rotatebox{180}{ee}};$z D x  A x  D x  D u  C u R u  D y  A y  A y ; z  \Square D v  A v  A v  D v  D x  C x  C x  R w  D w  A  w  :: 0 : \Square D y A y$
\end{small}\par

\paragraph{}
Zustand $inst\rightarrow anf$\\
Maschine muss Information über den folgenden Zustand ans Ende der Eingabe schreiben um ihn im nächsten Durchlauf der Prozedur mit dem jetzt aktuellen Zustand $s_2$, bzw $DAA$ fortzufahren.\\
Anschließend wird von $anf$ an begonnen und alle Markierungen werden gelöscht.


\begin{small}
\par
\paragraph{}
\raisebox{\depth}{\rotatebox{180}{ee}};$z D x  A x  D x  D u  C u R u  D y  A y  A y ; z  \Square D v  A v  A v  D v  D x  C x  C x  R w  D w  A  w  ::0: \Square D y A y$
\end{small}\par
\paragraph{}
nach $find\_last(l(branch(R \dots )))$ (Rechter branch) ist die neue vollständige Konfiguration:\\
\begin{small}
\raisebox{\depth}{\rotatebox{180}{ee}};$\Square D\Square A\Square D\Square D\Square C\Square R\Square D\Square A\Square A\Square ;\Square \Square D\Square A\Square A\Square D\Square D\Square C\Square C\Square R\Square D\Square A\Square :: 0 : \Square D \Square A \Square A \Square D\Square D\Square C\Square C\Square R\Square D\Square A \Square$\\
\end{small}\par
\paragraph{}
Das Band ist nun frei von Markierungen, hat den ersten Übergang der Maschine $M$ erfolgreich simuliert und ist wieder im Zustand $anf$. Nun laufen die letzten Schritte endlos ab und es wird eine Folge von 01010... auf das Band geschrieben.




\end{document}