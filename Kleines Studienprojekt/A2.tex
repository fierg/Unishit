%\documentclass[12pt,a4paper,onecolumn]{beamer}

\documentclass[10pt, xcolor=x11names]{beamer}
\usecolortheme{seagull}
\useoutertheme{infolines}
\usefonttheme[onlymath]{serif}
\setbeamertemplate{headline}[default]
\setbeamertemplate{navigation symbols}{}
\mode<beamer>{\setbeamertemplate{blocks}[rounded][shadow=true]}
\setbeamercovered{transparent}
\setbeamercolor{block body example}{fg=blue, bg=black!20}

\usepackage[utf8]{inputenc}
\usepackage[german]{babel}
\usepackage[]{csquotes}

\usepackage{amsmath}
\usepackage{tikz, wasysym}
\usetikzlibrary{automata,positioning}
%\usepackage{amsfonts}
%\usepackage{amssymb}
%\usepackage{makeidx}
%\usepackage{graphicx}
\author{Sven Fiergolla}
\title[Kleines Studienprojekt]{Eine universelle Turingmaschine mit zwei Zuständen/Symbolen}
\subtitle[short version]{Ein Paper von Claude E. Shannon}
\date{\today}
%\institute[Uni Trier]{Universität Trier}
%\logo{\includegraphics[scale=.25]{unilogo.pdf}}

\begin{document}
	\frame{\maketitle}
	%\tableofcontents

	\section{Einführung}
	\begin{frame}
		
	\frametitle{Einführung}
Formal definieren wir die Turingmaschine als Septupel $\mathbf{ M = (Q,\Sigma,\Gamma,q_0,\delta,\Square,F)} $
 wobei:

\begin{description}


 \item $\mathbf{ Q = }$ die endliche Zustandsmenge
 \item $\mathbf{ \Sigma = }$ das endliche Eingabealphabet
 \item $\mathbf{ \Gamma = }$ das endliche Bandalphabet und es gilt $\mathbf{\Sigma\subset\Gamma}$
 \item $\mathbf{ q_0 = }$ der Anfangszustand
 \item $\mathbf{ \delta = }$ die (partielle) Überführungsfunktion
 \item $\mathbf{ \square = }$ steht für das leere Feld (Blank)
 \item $\mathbf{F = }$ die Menge der akzeptierenden Endzustände

\end{description}	
	
	
	\end{frame}
	
	\frame{\frametitle{Beispiel}
	
	\begin{tikzpicture}[shorten >=1pt,node distance=2cm,on grid,auto]
   \node[state,initial] (0) {$q_0$};
   \node[state] (1) [right=of 0] {$q_1$};
   \node[state] (2) [right=of 1] {$q_2$};
   \node[state] (3) [above=of 2] {$q_3$};
   \node[state] (4) [right=of 3] {$q_4$};
   % push this node further to the right
   \node[state,accepting] (5) [right=3cm of 2] {$q_5$};

   \path[->]
    (0) edge                    node {$B\:B\:R$} (1)
    (1) edge [loop above]       node {$\begin{matrix}0\:0\:R\\1\:1\:R\end{matrix}$} (1)
    (1) edge [loop below]       node {$1\:1\:R$} (1)
    (1) edge                    node {$E\:E\:L$} (2)
    (2) edge [in=30, out=60, loop]       node {$*\:*\:L$} (2)
    (2) edge                    node {$1\:*\:R$} (3)
    (3) edge [loop above]       node {$*\:*\:R$} (3)
    (3) edge                    node {$E\:E\:R$} (4)
    (4) edge [loop above]       node {$0\:0\:R$} (4)
    (4) edge [loop right]       node {$1\:1\:R$} (4)
    (4) edge                    node {$\Box\:1\:L$} (5)
    (5) edge [loop right]       node {$\begin{matrix}0\:0\:L\\1\:1\:L\end{matrix}$} (5)
    (5) edge                    node {$E\:E\:L$} (2);
\end{tikzpicture}
	
}
	\section{universelle Turingmaschinen}
	\frame{\frametitle{Universelle Turingmaschinen}
		Formal ist eine universelle Turingmaschine eine Maschine $UTM$, die eine Eingabe $w|x$ liest. Das Wort $w$ ist hierbei eine  die Beschreibung einer Turingmaschine $M_{w}$, die zu einer bestimmten Funktion mit Eingabe $x$ die Ausgabe berechnet. $UTM$ simuliert also das Verhalten von $M_{w}$ mit Hilfe der Funktionsbeschreibung $w$ und der Eingabe $x$.
	}
	
	
	\section{Turingaschine mit zwei Zuständen}
	\frame{\frametitle{Konstruktion}
	Turingmaschine $A$: $ A_1,A_2,...,A_m \in$ $\Sigma_A$ die Symbole und $q_1,q_2,...q_n \in Q_A$ die Zustände der Maschine.
	Maschine B besitzt:
\begin{itemize}
	\item elementare Symbolen von Maschine $A$:  $B_1,B_2,...,B_m \in \Sigma_B$
	\item $m\cdot n \cdot 2 \cdot 2$ neue Symbole, welche Informationen über den Zustand und den Status der \textit{bouncing operation} speichern: $B_{m,n,x,y} \in \Sigma_B$
	\begin{itemize}
	\item $m =$ Symbole von $A, |\Sigma_A|$
	\item $n =$ Zustände von $A, |Q_A|$
	\item $x = +$ oder $-$ ob der Zustand des letzten Feldes in diese Feld übertragen wird oder aus diesem Feld stammt 
	\item $y = R$ oder $L$  ob die Information in das rechte oder linke Feld übertragen wird.
	 
	\end{itemize}
\end{itemize}	
	 
	 Insgesammt besitzt Maschine $B$ also $m+4mn$ Symbole.  
	}
	
	\frame{\frametitle{Zustände}
	Die Zustände von Maschine $B$ werden $\alpha$ und $\beta$ heißen.\\
	\medskip
	Um die Information des aktuellen Zustands nach bearbeiten eines Symbols in der nächsten Zelle zur Verfügung zu haben, auch wenn die $TM$ $B$ nur zwei Zustände hat, wird diese in den Symbolen gespeichert (Index $n$) und über die sogenannte \textit{bouncing operation} in die nächste Zelle übertragen.
	}
	
	\frame{\frametitle{Übergänge}
	\begin{center}
	\begin{tabular}{c c|c||c|c|c}

\textbf{Nr.}&\textbf{Symbol} & \textbf{Zustand $\Rightarrow$} & \textbf{Symbol} & \textbf{Zustand} &\textbf{Richtung} \\
\hline
(1) & $B_i$ & $\alpha$  & $B_{i,1,-,R}$  & $\alpha$ & $R$\\
\hline
(2) & $B_i$ & $\beta$  & $B_{i,1,-,L}$  & $\alpha$ & $L$\\
\hline
(3) & $B_{i,j,-,x}$ & $\alpha$ oder $\beta$  & $B_{i,(j+1),-,x}$  & $\alpha$ & $x\in \{ R,L \}$\\
\hline
(4) & $B_{i,j,+,x}$ & $\alpha$ oder $\beta$  & $B_{i,(j-1),+,x}$  & $\beta$ & $x\in \{ R,L \}$\\
\hline
(5) & $B_{i,1,+,x}$ & $\alpha$ oder $\beta$  & $B_i$  & $\alpha$ & $x\in \{ R,L \}$\\


\end{tabular}
\\
\bigskip
zusätzlich erhält Maschine $B$ für jeden Übergang in $A :$\\
\bigskip
(6) $\delta(A_i,q_j) \to (A_k,q_l,{R \atop L}) \Rightarrow \delta(B_{i,j,-,x},\alpha) \to (B_{k,l,+,{R \atop L} },{\beta \atop \alpha},{R \atop L})$\\


\end{center}
}
\section{Beispiel}
	\frame{\frametitle{Beispiel Maschine $A$}
	\begin{center}
	
	
	Maschine $A$:\\
	\bigskip
	$...|\underbrace{A_3}|A_{13}|...$
	
	\bigskip
		\begin{tikzpicture}[shorten >=1pt,node distance=3cm,on grid,auto]
   
   \node[state] (1) [right=of 0] {$q_7$};
   \node[state] (2) [right=of 1] {$q_4$};
   

   \path[->]
    (1) edge                    node {$A_3\:A_8\:R$} (2);

\end{tikzpicture}



\bigskip
$...|A_8|\underbrace{A_{13}}|...$
\end{center}
	}
	
	
	%TODO: nach und nach einblenden
	\frame{\frametitle{Beispiel Maschine $B$}
	\begin{tabular}{c||c c}
	\textbf{Bandinhalt} & \textbf{Übergangsfunktion} & \textbf{Gleichung}\\ \hline
	\\
	$...|\underbrace{B_{3,7,-,x}}|B_{13}|...$ &
	$\delta(B_{3,7,-,x},\alpha) = (B_{8,4,+,R},\beta,R)$ &  $(6)$\\ 
	\hline
	$...|B_{8,4,+,R}|\underbrace{B_{13}}|...$ & $\delta(B_{13},\beta) = (B_{13,1,-,L},\alpha,L)$ & $(2)$\\
	\hline
	$...|\underbrace{B_{8,4,+,R}}|B_{13,1,-,L}|...$ & $\delta(B_{8,4,+,R},\alpha) = (B_{8,3,+,R},\beta,R)$ & $(4)$\\
	\hline
	$...|B_{8,3,+,R}|\underbrace{B_{13,1,-,L}}|...$ &$\delta(B_{13,1,-,L},\beta) = (B_{13,2,-,L},\alpha,L)$  & $(3)$\\
	\hline
	$...|\underbrace{B_{8,3,+,R}}|B_{13,2,-,L}|...$ & $\delta(B_{8,3,+,R},\alpha) = (B_{8,2,+,R},\beta,R)$ & $(4)$\\
	\hline
	$...|B_{8,2,+,R}|\underbrace{B_{13,2,-,L}}|...$ & $\delta(B_{13,2,-,L},\beta) = (B_{13,3,-,L},\alpha,L)$ & $(3)$\\
	\hline
	$...|\underbrace{B_{8,2,+,R}}|B_{13,3,-,L}|...$ & $\delta(B_{8,2,+,R},\alpha) = (B_{8,1,+,R},\beta,R)$ & $(4)$\\
	\hline
	$...|B_{8,1,+,R}|\underbrace{B_{13,3,-,L}}|...$ & $\delta(B_{13,3,-,L},\beta) = (B_{13,4,-,L},\alpha,L)$ & $(3)$\\
	\hline
	$...|\underbrace{B_{8,1,+,R}}|B_{13,4,-,L}|...$ & $\delta(B_{8,1,+,R},\alpha) = (B_{8},\alpha,R)$ & $(5)$\\
	\hline
	$...|B_{8}|\underbrace{B_{13,4,-,L}}|B_{x}|...$ & $\dots$ & $(6)$\\
	\end{tabular}
	}
	
	\section{Unmöglichkeit einer universellen Turingmaschine mit einem Zustand}
	\frame{\frametitle{die Quadratwurzel von $2$ mit nur einem Zustand berechnen}
	Beweis per () von Schannon:\\
	Annahme: es existiert eine universelle Turingmaschine mit nur einem Zustand.\\
	\bigskip
	$\sqrt{2}$ ist turingberechenbar $\Rightarrow$ eine $UTM$ kann $\sqrt{2}$ berechnen $\Rightarrow$ eine $TM$ mit einem Zustand kann $\sqrt{2}$ berechnen.
	}
	
	
	\frame{\frametitle{die Quadratwurzel von $2$ mit nur einem Zustand berechnen}
	Fall $1:$ doppelt unendliches Band
	\begin{itemize}
	\item $1.1:$ Lesekopf liest $\Square \Rightarrow$ Lesekopf bleibt im $\Square$-Bereich
	\item $1.2:$ Lesekopf verlässt $\Square$
	\begin{itemize}
	\item $1.2.1:$ Lesekopf verlässt $\Square$ nach Links
	\begin{itemize}
	\item $1.2.1.1$ linke unendliche Seite des Bandes wird nicht betreten
	\item $1.2.1.2$ linke unendliche Seite des Bandes wird betreten
	\end{itemize}
	\item $1.2.2:$ Lesekopf verlässt $\Square$ nach Rechts
	\end{itemize}
	\end{itemize}

	
	%1.1
	$1.1$ Die $TM$ wird nie mehr als ein $\Square$ der Eingabe verändern $\Rightarrow$ das Eingabeband ist nur auf einem endlichen Teil beschrieben $\Rightarrow$ das Band kann nach der Bearbeitung nicht $\sqrt{2}$ enthalten.
	
	$1.2.1.1$ Die $TM$ betritt nur eine Seite des Bandes $\rightarrow$ wird in Fall $2$ behandelt
	
	$1.2.1.2$ Die $TM$ geht unendlich weit nach Links $\Rightarrow$ linke Seite des Bandes wird mit konstantem Symbol beschrieben und rechte unendliche Seite des Bandes nie betreten $\Rightarrow$ Band kann nach der Bearbeitung nicht $\sqrt{2}$ enthalten.
	
	$1.2.2$ analog zu $1.2.1$.	
	}
	
	\frame{\frametitle{die Quadratwurzel von $2$ mit nur einem Zustand berechnen}
	Fall $2:$ einseitig unendliches Band
	Annahe: Band ist rechts der Eingabe unendlich.
	Beweishilfe: \enquote{\textit{reflection number}}
		
	}
	
	
	
	
\end{document}