%\documentclass[12pt,a4paper,onecolumn]{beamer}

\documentclass[10pt, xcolor=x11names]{beamer}
\usecolortheme{seagull}
\useoutertheme{infolines}
\usefonttheme[onlymath]{serif}
\setbeamertemplate{headline}[default]
\setbeamertemplate{navigation symbols}{}
\mode<beamer>{\setbeamertemplate{blocks}[rounded][shadow=true]}
\setbeamercovered{transparent}
\setbeamercolor{block body example}{fg=blue, bg=black!20}

\usepackage[utf8]{inputenc}
\usepackage[german]{babel}
\usepackage[T1]{fontenc}
% needed for the matrix envirionment
\usepackage{amsmath}
\usepackage{tikz, wasysym}
\usetikzlibrary{automata,positioning}
%\usepackage{amsfonts}
%\usepackage{amssymb}
%\usepackage{makeidx}
%\usepackage{graphicx}
\author{Sven Fiergolla}
\title[Kleines Studienprojekt]{Eine universelle Turingmaschine mit zwei Zuständen}
\subtitle[short version]{Ein Paper von Claude E. Shannon}
\date{\today}
%\institute[Uni Trier]{Universität Trier}
%\logo{\includegraphics[scale=.25]{unilogo.pdf}}

\begin{document}
	\frame{\maketitle}
	
	\begin{frame}
	
	\section{Einführung}
	Einführung:
	
	\begin{tikzpicture}[shorten >=1pt,node distance=2cm,on grid,auto]
   \node[state,initial] (0) {$q_0$};
   \node[state] (1) [right=of 0] {$q_1$};
   \node[state] (2) [right=of 1] {$q_2$};
   \node[state] (3) [above=of 2] {$q_3$};
   \node[state] (4) [right=of 3] {$q_4$};
   % push this node further to the right
   \node[state] (5) [right=3cm of 2] {$q_5$};

   \path[->]
    (0) edge                    node {$B\:B\:R$} (1)
    (1) edge [loop above]       node {$\begin{matrix}0\:0\:R\\1\:1\:R\end{matrix}$} (1)
    (1) edge [loop below]       node {$1\:1\:R$} (1)
    (1) edge                    node {$E\:E\:L$} (2)
    (2) edge [in=30, out=60, loop]       node {$*\:*\:L$} (2)
    (2) edge                    node {$1\:*\:R$} (3)
    (3) edge [loop above]       node {$*\:*\:R$} (3)
    (3) edge                    node {$E\:E\:R$} (4)
    (4) edge [loop above]       node {$0\:0\:R$} (4)
    (4) edge [loop right]       node {$1\:1\:R$} (4)
    (4) edge                    node {$\Box\:1\:L$} (5)
    (5) edge [loop right]       node {$\begin{matrix}0\:0\:L\\1\:1\:L\end{matrix}$} (5)
    (5) edge                    node {$E\:E\:L$} (2);
\end{tikzpicture}
	
	\end{frame}
\end{document}