\documentclass[12pt, a4paper]{article}
\usepackage{amssymb}
\usepackage[utf8]{inputenc}
\usepackage[]{csquotes}
\usepackage[german]{babel}
\usepackage{graphicx}

\begin{document}

\paragraph{Sven Fiergolla 1252732}
\section*{3. Übung, Aufgabe 2}
\paragraph{Ergebnisse der Messung}
Auf einem Rechner kam ich auf einen maximalen Durchsatz von $0,1 \frac{mbit}{s}$ und bei einer Messung zwischen meinem Rechner und einem raspberryPi im gleichen Netzwerk auf $0,03 \frac{mbit}{s}$, lokal macht es keinen großen Unterschied ob Nachrichten einzelnd gesendet werden oder bis zu n (habe mit max 5 getestet) gleichzeitig auf eine Bestätigung warten.\par
\paragraph{Diskussion der Ergebnisse}
Habe lange versucht es zu optimieren und ich habe einen viel besseren Durchsatz erreicht, wenn ich deutlich größere Nachrichten verschickt habe als nur einzelne $int$, lag jedoch in allen Versuchen \textbf{weit} hinter dem theoretisch möglichen Durchsatz des Netzwerks/Speichers.\\
In der jetzt vorliegenden Version werden jedoch wieder nur einzelne Zeichen versendet und der Durchsatz ist fast erschreckend niedrig.\par



\paragraph{Hinweis}
Client muss vor dem Host gestartet werden!


\end{document}